% Template for ICASSP-2018 paper; to be used with:
%          spconf.sty  - ICASSP/ICIP LaTeX style file, and
%          IEEEbib.bst - IEEE bibliography style file.
% --------------------------------------------------------------------------
\documentclass{article}
\usepackage{spconf,amsmath,graphicx}
\usepackage{lineno,hyperref}
\usepackage{subfig}
\usepackage{amssymb}
\usepackage{bm}
\usepackage{booktabs}
\usepackage{multirow}
%\usepackage{diagbox}
\usepackage{amsthm}
\usepackage{graphicx}
\usepackage{algorithmic}
\usepackage{array}
\usepackage{graphicx} % Allows including images
\usepackage{booktabs} % Allows the use of \toprule, \midrule and \bottomrule in tables
\usepackage[english]{babel}
\usepackage{algorithm,algorithmic} 
\usepackage{amsmath}
\usepackage{amssymb}
\DeclareMathOperator*{\argmax}{argmax}
% Example definitions.
% --------------------
\def\x{{\mathbf x}}
\def\L{{\cal L}}

% Title.
% ------
\title{CMVTF: }
%
% Single address.
% ---------------
\name{Yipeng Liu,Sixing Zeng \thanks{Thanks to XYZ agency for funding.}}
\address{School of Information and Communication Engineering / Center for robotics, \\University of Electronic Science and Technology of China (UESTC), Chengdu, China}

%
% For example:
% ------------
%\address{School\\
%	Department\\
%	Address}
%
% Two addresses (uncomment and modify for two-address case).
% ----------------------------------------------------------
%\twoauthors
%  {A. Author-one, B. Author-two\sthanks{Thanks to XYZ agency for funding.}}
%	{School A-B\\
%	Department A-B\\
%	Address A-B}
%  {C. Author-three, D. Author-four\sthanks{The fourth author performed the work
%	while at ...}}
%	{School C-D\\
%	Department C-D\\
%	Address C-D}
%
\begin{document}
%\ninept
%
\maketitle
%
\begin{abstract}
Coupled Matrix-Vector Tensor Factorization(CMVTF)  is proposed for fusing a low-spatial-resolution hyperspectral image(HSI) and a high-spatial-resolution multispectral image(MSI) to produce a super-resolution image (SRI) with high spatial and spectral resolutions.In the proposed CMVTF method,We consider the HSI and MSI as a three-dimensional tensor and can be decomposed into a sum of several component tensor , and each component tensor is from the outer produt of a vector(endmember) and a matrix(corresponding abundances). This coupled tensor factorization model is consistent with linear spectral mixture model and experiments demonstrate the superiority of the proposed CMVTF method over current state-of-the-art HSI-MSI fusion approaches.
\end{abstract}
%
\begin{keywords}
One, two, three, four, five
\end{keywords}
%
\section{Introduction}
\label{sec:intro}

These guidelines include complete descriptions of the fonts, spacing, and
related information for producing your proceedings manuscripts. Please follow
them and if you have any questions, direct them to Conference Management
Services, Inc.: Phone +1-979-846-6800 or email
to \\\texttt{papers@2018.ieeeicassp.org}.

\section{Notations and Preliminaries}
\label{sec:pagestyle}

Firstly we would like to give the notations to be used. A vector, a matrix, and a tensor are written as  $  \mathbf{a} $,  $  \mathbf{A} $, and $\mathcal{A}$, respectively. A $ N $-order tensor $ \mathcal{A} \in \mathbb{R}^{ I_{1} \times\cdots\times I_{N}}$, and $I_{n}$ is the dimensional size. The vectorization of tensor $\mathcal{A}$ is denoted as $\operatorname {Vec}({\mathcal{A}})$. 
$\langle{\mathcal A, \mathcal B}\rangle =\langle \operatorname{Vec}(\mathcal{A}),\operatorname{Vec}(\mathcal{B})\rangle  $ denotes the tensor inner product. The Frobenius norm of $\mathcal{ A} $ is denoted as $\lVert{\mathcal A}\rVert_{\text F}$ =$\langle{\mathcal A,\mathcal A}\rangle^\frac{1}{2} $. The Kronecker product of two tensors can be denoted as $\mathcal{C}=\mathcal{A}\otimes\mathcal{B}\in\mathbb{R}^{I_{1}J_{1}\times\cdots\times I_{N}J_{N}}$, where $\mathcal{A}\in\mathbb{R}^{I_{1}\times\cdots\times I_{N}}$, $\mathcal{B}\in \mathbb{R}^{J_{1}\times\cdots\times J_{N}}$. For $M$ same tensors, the Kronecker product can be defined as $\mathcal{C}=\mathcal{A}_{1}\otimes\cdots\otimes\mathcal{A}_{M}=\otimes_{m=1}^{M}\mathcal{A}_{m}\in\mathbb{R}^{I_{1}^{M}\times\cdots\times I_{N}^{M}}$, with $\mathcal{A}\in\mathbb{R}^{I_{1}\times\cdots\times I_{N}}$.
The Hadamard product of two tensors is defined as $(\mathcal{A} \odot \mathcal{B})_{i_1, \cdots , i_N}=\mathcal{A}_{i_1, \cdots , i_N}\mathcal{B}_{i_1, \cdots , i_N}$, where $ {\mathcal{A}}_{i_1, \cdots , i_N} $ and $ {\mathcal{B}}_{i_1, \cdots , i_N} $ are the entries of $ {\mathcal{A}} $ and $ {\mathcal{B}} $. 
The operator $\operatorname{reshape}$ turns a matrix or a vector to a tensor, e.g., we can obtain a tensor $\mathcal{X}$ by the operating $\operatorname{reshape}(\mathbf{X}, I_{1}, I_{2},I_{3})$ from a matrix $\mathbf{X}\in \mathbb{R}^{I_{1}I_{2}\times I_{3}}$ or $\operatorname{reshape}(\mathbf{x}, I_{1}, I_{2},I_{3})$ from a vector $\mathbf{x}\in \mathbb{R}^{I_{1}I_{2}I_{3}}$. The mide-$n$ tensor-matrix product can be presented as $\mathcal{C}=\mathcal{A}\times_{n}\mathbf{B}$, where $\mathcal{A}\in \mathbb{R}^{I_{1} \times\cdots\times I_{N}}$, $\mathbf{B}\in \mathbb{R}^{J\times I_{n}}$, $\mathcal{C}\in \mathbb{R}^{I_{1} \times\cdots\times I_{n-1}\times J\times I_{n+1} \times I_{N}}$.

\textbf{Definition 1} (\emph{\textbf{Mode-k unfolding}})~\cite{cichocki2014tensor}
\emph{For an  $N$-order tensor $\mathcal{A} \in \mathbb{R}^{\emph{I}_{1} \times\cdots\times \emph{I}_{N}}$, its mode-k unfolding is defined as a matrix }
\begin{equation}
\textbf{A}_{[k]} \in \mathbb{R}^{I_k\times I_{1} \cdots I_{k-1}I_{k+1}\cdots I_N}\nonumber
\end{equation}
\emph{with entries}
\begin{equation}
a_{i_1i_2\cdots i_N}=\textbf{A}_{[k]}( i_k,\overline{i_{1}\cdots i_{k-1}i_{k+1}\cdots i_N}) \nonumber.
\end{equation}


\textbf{Definition 2} (\emph{\textbf{Tensor contracted  product}})~\cite{bibid}
\emph{For multiway arrays} $\mathcal{A} \in \mathbb{R}^{\emph{I}_1\times \cdots\times \emph{I}_N}$ \emph{and} $\mathcal{B} \in \mathbb{R}^{\emph{J}_1\times\cdots\times \emph{J}_L}$, with some common mode such as $I_{1}=J_{1},\cdots I_{n}=J_{n}$. \emph{The contracted product} $\mathcal{C} \in \mathbb{R}^{\emph{I}_{n+1}\times \cdots\times \emph{I}_N\times \emph{J}_{n+1}\times \cdots\times \emph{J}_L}$ \emph{can be written as}
\begin{equation}
\mathcal{C}=<\mathcal{A},\mathcal{B}>_n\nonumber
\end{equation}
\emph{	with entries}
\begin{eqnarray}
&\quad& c_{i_{n+1},\cdots,i_N,j_{n+1},\cdots,j_L}\nonumber
=\sum_{i_1=1,\cdots, i_n=1}^{I_1,\cdots,I_n}a_{i_1,\cdots,i_N}b_{j_1,\cdots,j_L}\nonumber.
\end{eqnarray}	

\textbf{Definition 3} (\emph{\textbf{Tensor in Graph formats}})~\cite{bibid}
\emph{Given any graph} $G=(\mathbb{V},\mathbb{E})$, \emph{where} $\mathbb{V}=\{1,\cdots,N\}$ \emph{and} $\mathbb{E}=\{\{i_{1},j_{1}\},\cdots,\{i_{M},j_{M}\}\}$ \emph{are the set of} $N$ \emph{vertices and} $M$ \emph{edges, respectively. $\hat{\mathbb{E}}=\{(i_{1},j_{1}),\cdots,(i_{M},j_{M})\}$ are the edges for arbitrary directions. For each} $n\in\mathbb{V}$, \emph{the tensor product space is defined as:}
\begin{equation*}
(\otimes_{m\in \text{IN}(n)}\mathbb{E}_{m})\otimes\mathbb{V}_{n}\otimes(\otimes_{m\in \text{OUT}(n)}\mathbb{E}_{m}^{*})
\end{equation*}
\emph{where} $\mathbb{E}_{m}^{*}$ \emph{is the dual space of} $\mathbb{E}_{m}$,  $\text{IN}(i)=\{j\in\{1,\cdots,M\}:(j,i)\in\hat{\mathbb{E}}\}$, \emph{and} $\text{OUT}(i)=\{j\in\{1,\cdots,M\}:(i,j)\in\hat{\mathbb{E}}\}$.

\emph{The contraction map for tensor can be defined as:}
\begin{eqnarray}
K_{G}:&&\otimes_{n=1}^{N}((\otimes_{m\in \text{IN}(n)}\mathbb{E}_{m})\otimes\mathbb{V}_{n}\otimes(\otimes_{m\in \text{OUT}(n)}\mathbb{E}_{m}^{*}))\nonumber\\
&&\rightarrow \otimes_{n=1}^{N} \mathbb{V}_{n}\nonumber
\end{eqnarray}
\emph{Contracting along all edges, we can obtain a tensor in}
$\mathbb{V}_{1}\otimes\cdots\otimes\mathbb{V}_{N}$.

\textbf{Definition 4} (\emph{\textbf{Complete-graph tensor network}})
\emph{For a multiway array} $\mathcal{A} \in \mathbb{R}^{I_{1} \times\cdots\times I_{N}}$, \emph{the complete-graph tensor network in graph is defined as}
\begin{equation}
\mathcal{A}=K_{G}(\otimes_{n=1}^{N}(\mathcal{U}_{n})\nonumber).
\end{equation}	
\emph{where all edges} $\mathbb{E}$ \emph{for complete graph are assigned to} $\text{IN}$, \emph{and} $\mathcal{U}_{n}=((\otimes_{m\in\text{IN}_{(n)}}\mathbb{E}_{m})\otimes\mathbb{V}_{n})\in\mathbb{R}^{r^{m}\times I_{n}}$,  $n=1, \cdots, N$ \emph{are the cores, and the rank of complete graph tensor network is defined by the weigh of edges.} \emph{The graphical of 4-order complete graph tensor network can be seen in Fig. \ref{fig:TT decomposition}}.

\textbf{Proposition 1} (\emph{\textbf{Edge removal}})~\cite{bibid} \emph{For an N-order tensor} $\mathcal{A}\in\mathbb{R}^{I_{1}\times I_{N}}$, \emph{which can be represented by the} 
$G=(\mathbb{V},\mathbb{E})$  \emph{with} $N$ \emph{vertices and} $M$ \emph{edges. Let} $G_{r}=(r_{1},\cdots,r_{M})$ \emph{are the weight of edges. Suppose that the edge} $e\in\mathbb{E}$\emph{ has weight} $r_{1}=1$ \emph{and} $G^{'}=(\mathbb{V},\mathbb{E}\setminus \{e\})$ \emph{is the graph obtained by removing the edge} $e$ from $G$ \emph{and suppose} $G^{'}$ \emph{has no isolated vertices. Then}
\begin{equation*}
\mathcal{A}=K_{G}\simeq K_{G^{'}}.
\end{equation*}

With edge removal, tensor train decomposition and tensor ring decomposition are special cases of complete-graph tensor network. An example of the equal conversion can be seen as Fig. (\ref{fig:edgeremoval}).

\section{CMVTF FOR DATA FUSION}
\label{sec:format}`

In this paper we consider hyperspectral image and multispectral image as a three-dimension tensor. Hyperspectral imae($\mathcal{X}\in \mathbb{R}^{I_H\times J_H\times  K} $) is low-spatial-resolution but high-spectral-resolution and multispectral($\mathcal{Y} \in \mathbb{R}^{I\times J\times  K_M} $) image is high-spatial-resolution but low-spectral-resolution . The aim of data fusion is to estimate a super-resolution image($\mathcal{Z} \in \mathbb{R}^{I\times J\times  K} $) that with high resolution both in spatial and spectral. $I,J$ and $K$ denote the dimensions of two spatial and one spectral coordinates,respectively.Obviously,$I_H \ll I , J_H \ll J $ and $K_M \ll K $ are satisfied by the spcetral and spatital resloutions of the two images.


\section{PAGE TITLE SECTION}
\label{sec:pagestyle}

The paper title (on the first page) should begin 1.38 inches (35 mm) from the
top edge of the page, centered, completely capitalized, and in Times 14-point,
boldface type.  The authors' name(s) and affiliation(s) appear below the title
in capital and lower case letters.  Papers with multiple authors and
affiliations may require two or more lines for this information. Please note
that papers should not be submitted blind; include the authors' names on the
PDF.

\begin{table}[htbp]
	\centering
	\begin{tabular}{lll}
		\toprule
		$\bf Algorithm $ : CNMVF algorithm  \\
		
		\midrule
		$ \bf Input: HSI,MSI,D1,D2,D3,\lambda $ \\
		$ \bf While $ not converged do\\
		1. $ A =	$ arg $ \min_A || \mathrm{Y_h}^{(1)}-S_{11}(D_1A)^T||_F^2+\lambda||\mathrm{Y_m}^{(1)}-S_{12}A^T||_F^2 $ \\
		
		2. $ B =$ arg $	\min_B || \mathrm{Y_h}^{(2)}-S_{21}(D_2B)^T||_F^2+\lambda||\mathrm{Y_m}^{(2)}-S_{22}B^T||_F^2 $\\
		3. $ C =$ arg $	\min_C || \mathrm{Y_h}^{(3)}-S_{31}C^T||_F^2+\lambda||\mathrm{Y_m}^{(3)}-S_{32}(D_3C)^T||_F^2 $\\
		\bf End while \\
		\bottomrule	
	\end{tabular}
\end{table}


\section{TYPE-STYLE AND FONTS}
\label{sec:typestyle}

To achieve the best rendering both in printed proceedings and electronic proceedings, we
strongly encourage you to use Times-Roman font.  In addition, this will give
the proceedings a more uniform look.  Use a font that is no smaller than nine
point type throughout the paper, including figure captions.

In nine point type font, capital letters are 2 mm high.  {\bf If you use the
smallest point size, there should be no more than 3.2 lines/cm (8 lines/inch)
vertically.}  This is a minimum spacing; 2.75 lines/cm (7 lines/inch) will make
the paper much more readable.  Larger type sizes require correspondingly larger
vertical spacing.  Please do not double-space your paper.  TrueType or
Postscript Type 1 fonts are preferred.

The first paragraph in each section should not be indented, but all the
following paragraphs within the section should be indented as these paragraphs
demonstrate.

\section{MAJOR HEADINGS}
\label{sec:majhead}

Major headings, for example, "1. Introduction", should appear in all capital
letters, bold face if possible, centered in the column, with one blank line
before, and one blank line after. Use a period (".") after the heading number,
not a colon.

\subsection{Subheadings}
\label{ssec:subhead}

Subheadings should appear in lower case (initial word capitalized) in
boldface.  They should start at the left margin on a separate line.
 
\subsubsection{Sub-subheadings}
\label{sssec:subsubhead}

Sub-subheadings, as in this paragraph, are discouraged. However, if you
must use them, they should appear in lower case (initial word
capitalized) and start at the left margin on a separate line, with paragraph
text beginning on the following line.  They should be in italics.

\section{PRINTING YOUR PAPER}
\label{sec:print}

Print your properly formatted text on high-quality, 8.5 x 11-inch white printer
paper. A4 paper is also acceptable, but please leave the extra 0.5 inch (12 mm)
empty at the BOTTOM of the page and follow the top and left margins as
specified.  If the last page of your paper is only partially filled, arrange
the columns so that they are evenly balanced if possible, rather than having
one long column.

In LaTeX, to start a new column (but not a new page) and help balance the
last-page column lengths, you can use the command ``$\backslash$pagebreak'' as
demonstrated on this page (see the LaTeX source below).

\section{PAGE NUMBERING}
\label{sec:page}

Please do {\bf not} paginate your paper.  Page numbers, session numbers, and
conference identification will be inserted when the paper is included in the
proceedings.

\section{ILLUSTRATIONS, GRAPHS, AND PHOTOGRAPHS}
\label{sec:illust}

Illustrations must appear within the designated margins.  They may span the two
columns.  If possible, position illustrations at the top of columns, rather
than in the middle or at the bottom.  Caption and number every illustration.
All halftone illustrations must be clear black and white prints.  Colors may be
used, but they should be selected so as to be readable when printed on a
black-only printer.

Since there are many ways, often incompatible, of including images (e.g., with
experimental results) in a LaTeX document, below is an example of how to do
this \cite{Lamp86}.

\section{FOOTNOTES}
\label{sec:foot}

Use footnotes sparingly (or not at all!) and place them at the bottom of the
column on the page on which they are referenced. Use Times 9-point type,
single-spaced. To help your readers, avoid using footnotes altogether and
include necessary peripheral observations in the text (within parentheses, if
you prefer, as in this sentence).

% Below is an example of how to insert images. Delete the ``\vspace'' line,
% uncomment the preceding line ``\centerline...'' and replace ``imageX.ps''
% with a suitable PostScript file name.
% -------------------------------------------------------------------------




% To start a new column (but not a new page) and help balance the last-page
% column length use \vfill\pagebreak.
% -------------------------------------------------------------------------
%\vfill
%\pagebreak

\section{COPYRIGHT FORMS}
\label{sec:copyright}

You must submit your fully completed, signed IEEE electronic copyright release
form when you submit your paper. We {\bf must} have this form before your paper
can be published in the proceedings.

\section{RELATION TO PRIOR WORK}
\label{sec:prior}

The text of the paper should contain discussions on how the paper's
contributions are related to prior work in the field. It is important
to put new work in  context, to give credit to foundational work, and
to provide details associated with the previous work that have appeared
in the literature. This discussion may be a separate, numbered section
or it may appear elsewhere in the body of the manuscript, but it must
be present.

You should differentiate what is new and how your work expands on
or takes a different path from the prior studies. An example might
read something to the effect: "The work presented here has focused
on the formulation of the ABC algorithm, which takes advantage of
non-uniform time-frequency domain analysis of data. The work by
Smith and Cohen \cite{Lamp86} considers only fixed time-domain analysis and
the work by Jones et al \cite{C2} takes a different approach based on
fixed frequency partitioning. While the present study is related
to recent approaches in time-frequency analysis [3-5], it capitalizes
on a new feature space, which was not considered in these earlier
studies."

\vfill\pagebreak

\section{REFERENCES}
\label{sec:refs}

List and number all bibliographical references at the end of the
paper. The references can be numbered in alphabetic order or in
order of appearance in the document. When referring to them in
the text, type the corresponding reference number in square
brackets as shown at the end of this sentence \cite{C2}. An
additional final page (the fifth page, in most cases) is
allowed, but must contain only references to the prior
literature.

% References should be produced using the bibtex program from suitable
% BiBTeX files (here: strings, refs, manuals). The IEEEbib.bst bibliography
% style file from IEEE produces unsorted bibliography list.
% -------------------------------------------------------------------------
\bibliographystyle{IEEEbib}
\bibliography{strings,refs}

\end{document}
